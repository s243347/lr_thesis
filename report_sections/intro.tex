\section{Introduction}

\subsection{Signal Preprocessing}
\subsubsection{Whittaker Smoothing vs. Asymmetric Least Squares (AsLS) for Spectroscopy}

\textbf{Whittaker smoothing} is a classic method for smoothing data, formulated as a minimization problem:
\begin{equation*}
	\min_z \sum_i (y_i - z_i)^2 + \lambda \sum_i (D^2 z_i)^2
\end{equation*}
where $y$ is the observed signal, $z$ is the smoothed signal and $D^2$ is the second difference operator. The 
first term is the data fit, adnd the second represents the penalty for non-smooth signals. $\lambda$ balances
 the 2 terms. Whittaker smoothing is linear, treats positive and negative residuals equally (symmetric), and 
 has a single trade-off parameter $\lambda$. The solution is closed-form.

\textbf{Asymmetric Least Squares (AsLS)} modifies only the data fidelity term by introducing weights:
\begin{equation*}
	\min_z \sum_i w_i (y_i - z_i)^2 + \lambda \sum_i (D^2 z_i)^2
\end{equation*}
where $w_i = p$ if $y_i > z_i$ and $w_i = 1-p$ if $y_i < z_i$, with $p$ typically close to zero. The weights 
depend on the current solution $z$, making the problem iterative and asymmetric. AsLS actively ignores peaks 
by down-weighting positive residuals, focusing on fitting the lower envelope of the signal---ideal for baseline 
removal in spectroscopy.

In AsLS, $\lambda$ is typically chosen by heuristics (e.g., $\lambda \sim 10^4$--$10^8$ for spectroscopy) 
and visual inspection. In summary, AsLS uses Whittaker as a subproblem but adds asymmetric, adaptive weighting,
making it more suitable for baseline correction in spectroscopy.

\subsubsection{Low-pass filter}

For removing high-frequency noise from Raman spectra, a Butterworth low-pass filter is used. 

\subsubsection{Normalization}

Min-max normalization rescales every spectrum to the same range, typically [0, 1], using the formula:
\begin{equation*}
	x' = \frac{x - x_{min}}{x_{max} - x_{min}}		
\end{equation*}
where $x$ is the original value, $x_{min}$ and $x_{max}$ are the minimum and maximum values of the spectrum,
 and $x'$ is the normalized value.

However, in Raman, peak intensity carries chemical information. Thus, using min-max normalization amplifies 
noise and baselinea artifacts. Instead, standard normalization (z-score normalization) is preferred:

\begin{equation*}
	x' = \frac{x - \mu}{\sigma}
\end{equation*}

